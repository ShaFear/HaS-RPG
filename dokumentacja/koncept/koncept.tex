\documentclass[a4paper,11pt,notitlepage]{article}
\usepackage[utf8]{inputenc}	
\usepackage[T1]{fontenc}

\usepackage{graphicx}
\usepackage[polish]{babel}

\author{Michał Jereczek}
\title{ Hide and seek RPG \\ {\small koncept}}
\date{\today}

\linespread{1.3}

\usepackage{indentfirst}

\begin{document}
\maketitle
\tableofcontents
\pagebreak

\section{Ogólne}
\subsection{Krótki opis}
Aplikacja na system operacyjny Android wynosząca zabawę w chowanego na nowy, odświeżony poziom. Gracze (conajmniej trzech) spotykają się w jednym miejsu, wśród nich zostaje wybrany jeden łowca, reszta zostaje ofiarami. Ofiary mają określony czas na ucieczkę, po którym łowca korzystając z aplikacji namierza ich lokalizację GPS i następuje pościg. Ofiary nie mogą się ruszać z swojej kryjówki, jednak aplikacja będzie pozwalała na dodatkowe manewry, ucieczki, zmyłki itp. Gra kończy się gdy łowca znajdzie wszystkie ofiary lub po określonym czasie (po upływie tego czasu łowca przegrywa).

\end{document}
