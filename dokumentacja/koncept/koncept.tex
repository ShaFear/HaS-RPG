\documentclass[a4paper,11pt,notitlepage]{article}
\usepackage[utf8]{inputenc}	
\usepackage[T1]{fontenc}

\usepackage{graphicx}
\usepackage[polish]{babel}

\author{Michał Jereczek}
\title{ Hide and seek RPG \\ {\small koncept}}
\date{\today}

\linespread{1.3}

\usepackage{indentfirst}

\begin{document}
\maketitle
\tableofcontents
\pagebreak

\section{Ogólne}
\subsection{Krótki opis}
Aplikacja na system operacyjny Android wynosząca zabawę w chowanego na nowy, odświeżony poziom. Gracze (conajmniej trzech) spotykają się w jednym miejsu, wśród nich zostaje wybrany jeden łowca, reszta zostaje ofiarami. Ofiary mają określony czas na ucieczkę, po którym łowca korzystając z aplikacji namierza ich lokalizację GPS i następuje pościg. Ofiary nie mogą się ruszać z swojej kryjówki, jednak aplikacja będzie pozwalała na dodatkowe manewry, ucieczki, zmyłki itp. Gra kończy się gdy łowca znajdzie wszystkie ofiary lub po określonym czasie (po upływie tego czasu łowca przegrywa).

\subsection{Podstawowy scenariusz rozgrywki}
\begin{enumerate}
\item
	\textbf{Gracz} tworzy nowe \textbf{lobby} (staje się \textbf{administratorem}).
	
\item
	\textbf{Gracze} dołączają do \textbf{lobby}.
	
\item
	\textbf{Administrator} może wystartować \textbf{grę}, gdy będzie co najmniej trzech \textbf{graczy}.
	
\item
	\textbf{Poprzez} losowanie jeden z \textbf{graczy} zostaje \textbf{łowcą}.
	
\item
	Pozostali \textbf{gracze} zostają \textbf{ofiarami}.
	
\item
	W ciągu minuty \textbf{ofiary} jak i \textbf{łowca} wybierają \textbf{postacie}, którymi będą grać.
	
\item
	\textbf{Ofiary} mają 15 minut na \textbf{ucieczkę} do dowolnej \textbf{kryjówki}, w tym czasie \textbf{łowca} nie może się ruszać (i powinien zamnkąć oczy, dopóki \textbf{ofiary} się nie oddalą).
	
\item
	\textbf{Łowca} wyrusza w poszukiwaniu \textbf{ofiar}, rozpoczyna się \textbf{rozgrywka}.
	
\item
	\textbf{Ofiary} jak i \textbf{łowca} mogą stosować specjalne umiejętności w trakcie \textbf{rozgrywki} definiowane przez \textbf{postać}, którą grają.
	
\item
	Gdy \textbf{łowca} znajdzie się w określonej odległości od \textbf{ofiary} następuje \textbf{potyczka}.
	
\item
	Standardowo \textbf{potyczka} polega na predykcji zamierzeń \textbf{łowcy} przez \textbf{ofiarę}. \textbf{Łowca} wybiera część \textbf{ciała} (\textbf{góra} lub \textbf{dół}), którą atakuje. \textbf{Ofiara} wybiera przewidywaną część \textbf{ciała}, jeżeli dobrze przewidzi 3 razy pod rząd to ma prawo do 3 minutowej \textbf{ucieczki}. Jeżeli nie przewidzi chociaż raz, zostaje \textbf{pokonanym}.
	
\item
	\textbf{Rozgrywka} toczy się dopóki wszystkie \textbf{ofiary} nie zostaną \textbf{pokonanymi} lub po upływie godziny od rozpoczęcia \textbf{rozgrywki}.
	
\item
	Zostaje pokazany \textbf{ranking} i graczom zostają przydzielone \textbf{punkty doświadczenia}.

	
\end{enumerate}

\end{document}
