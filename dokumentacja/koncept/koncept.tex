\documentclass[a4paper,11pt,notitlepage]{article}
\usepackage[utf8]{inputenc}	
\usepackage[T1]{fontenc}

\usepackage{graphicx}
\usepackage[polish]{babel}

\author{Michał Jereczek}
\title{ Hide and seek RPG \\ {\small koncept}}
\date{\today}

\linespread{1.3}

\usepackage{indentfirst}

\begin{document}
\maketitle
\tableofcontents
\pagebreak

\section{Wymagania}
\begin{itemize}
\item Bezpieczeństwo danych użytkownika
\item Obsługa błędów związanych z komunikacją
\item Możliwość rejestracji, logowania
\item Tworzenie i dołączanie do Lobby gry
(lobby - pokój w którym gracze się \"{}organizują\"{} przed rozgrywką)
\item Możliwość przeprowadzenia rozgrywki tak jak opisano w \"{}Podstawowym scenariuszu rozgrywki\"{}
\item Wersja bazowa gry nie musi zawierać elementów RPG (ewoluowania postaci oraz dodatkowych umiejętności), jednak mogą zostać dodane w przyszłości jako dodatkowa funkcjonalość.
\item Należy zauważyć, że w podstawowym scenariuszu rozgrywki wszystkie odniesienia do czasu są czysto teoretyczne, w praktyce będą wymagały skalibrowania tak aby gra nie nudziła.

\end{itemize}

\section{Ogólne}
\subsection{Krótki opis}
Aplikacja na system operacyjny Android wynosząca zabawę w chowanego na nowy, odświeżony poziom. Gracze (conajmniej trzech) spotykają się w jednym miejsu, wśród nich zostaje wybrany jeden łowca, reszta zostaje ofiarami. Ofiary mają określony czas na ucieczkę, po którym łowca korzystając z aplikacji namierza ich lokalizację GPS i następuje pościg. Ofiary nie mogą się ruszać z swojej kryjówki, jednak aplikacja będzie pozwalała na dodatkowe manewry, ucieczki, zmyłki itp. Gra kończy się gdy łowca znajdzie wszystkie ofiary lub po określonym czasie (po upływie tego czasu łowca przegrywa).

\subsection{Podstawowy scenariusz rozgrywki}
\begin{enumerate}
\item
	\textbf{Gracz} tworzy nowe \textbf{lobby} (staje się \textbf{administratorem}).
	
\item
	\textbf{Gracze} dołączają do \textbf{lobby}.
	
\item
	\textbf{Administrator} może wystartować \textbf{grę}, gdy będzie co najmniej trzech \textbf{graczy}. (jeżeli admin się wyloguje, to kolejny w kolejności gracz dostaje jego uprawnienia).
	
\item
	\textbf{Poprzez} losowanie jeden z \textbf{graczy} zostaje \textbf{łowcą}.
	
\item
	Pozostali \textbf{gracze} zostają \textbf{ofiarami}.
	
\item
	\textbf{Ofiary} mają 15 minut na \textbf{ucieczkę} do dowolnej \textbf{kryjówki}, w tym czasie \textbf{łowca} nie może się ruszać (i powinien zamnkąć oczy, dopóki \textbf{ofiary} się nie oddalą).
	
\item
	\textbf{Łowca} wyrusza w poszukiwaniu \textbf{ofiar}, rozpoczyna się \textbf{rozgrywka}.
	
\item
	\textbf{Ofiary} jak i \textbf{łowca} mogą stosować specjalne umiejętności w trakcie \textbf{rozgrywki} definiowane przez \textbf{postać}, którą grają.
	
\item
	Gdy \textbf{łowca} znajdzie się w określonej odległości od \textbf{ofiary} następuje \textbf{potyczka}.
	
\item
	Standardowo \textbf{potyczka} polega na predykcji zamierzeń \textbf{łowcy} przez \textbf{ofiarę}. \textbf{Łowca} wybiera część \textbf{ciała} (\textbf{góra} lub \textbf{dół}), którą atakuje. \textbf{Ofiara} wybiera przewidywaną część \textbf{ciała}, jeżeli dobrze przewidzi 3 razy pod rząd to ma prawo do 3 minutowej \textbf{ucieczki}. Jeżeli nie przewidzi chociaż raz, zostaje \textbf{pokonanym}.
	
\item
	\textbf{Rozgrywka} toczy się dopóki wszystkie \textbf{ofiary} nie zostaną \textbf{pokonanymi} lub po upływie godziny od rozpoczęcia \textbf{rozgrywki}.
	
\item
	Zostaje pokazany \textbf{ranking} i graczom zostają przydzielone \textbf{punkty doświadczenia}.
	
\end{enumerate}

\section{Łowca}
\subsection{Krótki opis}
Gra łowcą polega na namierzaniu ofiar i pokonywaniu ich w potyczkach. Głównym celem łowcy jest pokonanie wszystkich graczy w ciągu określonego czasu. Podstawowo łowca ma dostęp do informacji jak daleko znajduje się od każdej z ofiar, jeżeli znajdzie się w określonym zasięgu może dokonać potyczki z ofiarą.

\subsection{Statystyki}
\begin{enumerate}
\item
	\textbf{Poziom doświadczenia} - początkowo 1, ostatecznie 20, każdy kolejny poziom doświadczenia dodaje jeden punkt umiejętności, który można użyć do zwiększenia bądź odblokowania umiejętności. Na 5 poziomie można wybrać jedną z dwóch klas łowcy, na poziomie 15 jedną z dwóch subklas wcześniej wybranej klasy.

\item
	\textbf{Punkty doświadczenia} - liczba aktualne posiadanych punktów doświadczenia.
	
\item
	\textbf{Punkty awansu} - liczba określająca ile trzeba mieć punktów doświadczenia aby uzyskać kolejny poziom doświadczenia (na każdym kolejnym poziomie doświadczenia liczba potrzebnych punktów jest dwukrotnie większa).	
	
\item
	\textbf{Początkowy zasięg potyczki} - określa zasięg w jakim ofiara musi się znaleźć aby móc ją zaatakować, jest to zasięg, który stopniowo narasta wraz z każdą kolejną poległą ofiarą (w trakcie potyczki).
	
\item
	\textbf{Końcowy zasięg potyczki} - określa końcowy zasięg, do którego może wzrosnąć początkowy zasięg potyczki.
	
\item
	\textbf{Aktualny zasięg potyczki} - określa aktualny zasięg w danej rozgrywce.
	
\item 
	\textbf{Punkty umiejętności} - określa ilość punktów, które można wykorzystać do wydania na możliwe dla danej klasy umiejętności.
	
\end{enumerate}

\subsection{Umiejętności}
\subsubsection{Klasa podstawowa}
\begin{enumerate}
\item \textbf{Szybki start} - pasywne, maksymalnie 5 punktów do rozdania. Z każdym kolejnym punktem doświadczenia różnica między początkowym, a końcowym zasięgiem potyczki jest zmniejszana o 5\%. (standardowo różnica jest równa 50\% w ten sposób można zmniejszyć do 25\%).

\item \textbf{Nie uciekniesz!} - 5-minut przeładowania, maksymalnie 5 punktów do rozdania. Jeżeli ofiara znajduje się w końcowym zasięgu potyczki, można zablokować jej możliwośc specjalnej ucieczki. Przy pierwszym punkcie doświadczenia można używać tej umiejętności na ofiarach z poziomem doświadczenia co najmniej o 10 większym, z każdym kolejnym punktem liczba ta maleje o 2 (przy pięciu punktach umiejętność tą można stosować na graczach o conajmniej takim poziomie jak łowca.

\item \textbf{Szybsza mobilizacja} - pasywne, maksymalnie 6 punktów do rozdania. Z każdym kolejnym punktem czas dostępny przy początkowej ucieczce (st. 15 minut) jest zmniejszany o 6\% (maksymalnie do 36\%).
\end{enumerate}

\subsubsection{Klasa 1}
TODO
\subsubsection{Klasa 1a}
TODO
\subsubsection{Klasa 1b}
TODO

\subsubsection{Klasa 2}
TODO
\subsubsection{Klasa 2a}
TODO
\subsubsection{Klasa 2b}
TODO

\section{Ofiara}
\subsection{Krótki opis}
Gra ofiarą polega na ukryciu się przed łowcą, głównym manewrem ofiary jest tak zwana \textbf{ucieczka}, która polega na tym, że przez określony czas ofiara może się przemieszczać, gdy czas się skończy jej lokalizacja GPS zostaje zapisana i ofiara nie może się z tamtąd ruszać do momentu kolejnej ucieczki. Ponadto standardowo ofiara może wyczuć, że łowca jest w pobliżu i jeżeli posiada specjalne umiejętności wykonać jakiś ruch.

\subsection{Statystyki}
\begin{enumerate}
\item
	\textbf{Poziom doświadczenia} - początkowo 1, ostatecznie 20, każdy kolejny poziom doświadczenia dodaje jeden punkt umiejętności, który można użyć do zwiększenia bądź odblokowania umiejętności. Na 5 poziomie można wybrać jedną z trzech klas ofiary, na poziomie 15 jedną z dwóch subklas wcześniej wybranej klasy.

\item
	\textbf{Punkty doświadczenia} - liczba aktualne posiadanych punktów doświadczenia.
	
\item
	\textbf{Punkty awansu} - liczba określająca ile trzeba mieć punktów doświadczenia aby uzyskać kolejny poziom doświadczenia (na każdym kolejnym poziomie doświadczenia liczba potrzebnych punktów jest dwukrotnie większa).	
	
\item
	\textbf{Zasięg wyczuwania łowcy} - określa odległość z jakiej ofiara wyczuwa łowcę (po osiągnięciu granicznej wielkości ofiara widzi dokładną odległość od łowcy w czasie rzeczywistym). 
	
\item 
	\textbf{Punkty umiejętności} - określa ilość punktów, które można wykorzystać do wydania na możliwe dla danej klasy umiejętności.
\end{enumerate}

\subsection{Umiejętności}
\subsubsection{Klasa podstawowa}
\begin{enumerate}
\item \textbf{Ucieczka} - standardowo ustawiona na 1, dwie na rozgrywkę, 5-minutowe przeładowanie, maksymalnie 5 punktów do rozdania ( w sumie 6). Można użyć w dowolnym momencie w rozgrywce, na pierwszym poziomie czas ucieczki trwa 2 minuty, każdy kolejny zwiększa o pół minuty.

\item \textbf{Zwiększona ostrożność} - standardowo ustawiona na 1, 10-minutowe przeładowanie, czas trwania 1 minuta, 6 punktów do rozdania. Tymczasowo zwiększa zasięg wyczuwania łowcy dwukrotnie, każdy kolejny punkt tej umiejętności zmniejsza czas przeładowania o minutę (można skrócić maksymalnie do 5 minut).

\item \textbf{Łatwiejsza potyczka} - standardowo ustawiona na 1, raz na rozgrywkę, 6 punktów do rozdania. Jeżeli w ciągu czasu trwania umiejętności zostanie się zaatakowanym przez łowcę, to potyczka zostaje zrównana do jednej tury (50\% szans na ucieczkę).  Każdy punkt zwiększa czas umiejętności o pół minuty (z pierwszym punktem jest to minuta).

\end{enumerate}

\subsubsection{Klasa 1}
TODO
\subsubsection{Klasa 1a}
TODO
\subsubsection{Klasa 1b}
TODO

\subsubsection{Klasa 2}
TODO
\subsubsection{Klasa 2a}
TODO
\subsubsection{Klasa 2b}
TODO

\subsubsection{Klasa 3}
TODO
\subsubsection{Klasa 3a}
TODO
\subsubsection{Klasa 3b}
TODO

\end{document}
